\documentclass[a4paper,14pt]{extarticle}
\usepackage{geometry}
\usepackage{fontspec}
\setmainfont[
  Ligatures=TeX,
  Extension=.otf,
  BoldFont=cmunbx,
  ItalicFont=cmunti,
  BoldItalicFont=cmunbi,
]{cmunrm}
\setsansfont[
  Ligatures=TeX,
  Extension=.otf,
  BoldFont=cmunsx,
  ItalicFont=cmunsi,
]{cmunss}
\usepackage[english,russian]{babel}
\usepackage{amsmath}
\usepackage{amsthm}
\usepackage{amssymb}
\usepackage{fancyhdr}
\usepackage{setspace}
\usepackage{graphicx}
\usepackage{colortbl}
\usepackage{tikz}
\usepackage{pgf}
\usepackage{subcaption}
\usepackage{listings}
\usepackage{indentfirst}
\usepackage[colorlinks,citecolor=blue,linkcolor=blue,bookmarks=false,hypertexnames=true, urlcolor=blue]{hyperref}
\usepackage{indentfirst}
\usepackage{mathtools}
\usepackage{booktabs}
\usepackage[flushleft]{threeparttable}
\usepackage{tablefootnote}

\usepackage{chngcntr} % нумерация графиков и таблиц по секциям
\counterwithin{table}{section}
\counterwithin{figure}{section}

\usepackage[backend=biber, citestyle=authoryear]{biblatex}
\addbibresource{bibliography.bib}
\DeclareFieldFormat{labelnumberwidth}{#1\adddot}
\setlength{\biblabelsep}{5pt}

\makeatletter
\renewcommand{\@biblabel}[1]{#1.} % Заменяем библиографию с квадратных скобок на точку:
\makeatother

\geometry{left=2.5cm}% левое поле
\geometry{right=1.5cm}% правое поле
\geometry{top=1.5cm}% верхнее поле
\geometry{bottom=1.5cm}% нижнее поле
\renewcommand{\baselinestretch}{1.5} % междустрочный интервал

\renewcommand{\theenumi}{\arabic{enumi}}% Меняем везде перечисления на цифра.цифра
\renewcommand{\labelenumi}{\arabic{enumi}}% Меняем везде перечисления на цифра.цифра
\renewcommand{\theenumii}{.\arabic{enumii}}% Меняем везде перечисления на цифра.цифра
\renewcommand{\labelenumii}{\arabic{enumi}.\arabic{enumii}.}% Меняем везде перечисления на цифра.цифра
\renewcommand{\theenumiii}{.\arabic{enumiii}}% Меняем везде перечисления на цифра.цифра
\renewcommand{\labelenumiii}{\arabic{enumi}.\arabic{enumii}.\arabic{enumiii}.}% Меняем везде перечисления на цифра.цифра

\DeclareMathOperator{\chr}{char}

\begin{document}
\begin{titlepage}
\newpage

{\setstretch{1.0}
\begin{center}
Федеральное государственное автономное образовательное учреждение высшего образования «Национальный исследовательский университет «Высшая школа экономики»
\\
\bigskip
Факультет компьютерных наук \\
Основная образовательная программа \\
Прикладная математика и информатика \\
\end{center}
}

\vspace{8em}

\begin{center}
{\Large КУРСОВАЯ РАБОТА}\\
\textsc{\textbf{
Исследовательский проект на тему
\linebreak
"Коммутативная алгебра"}}
\end{center}

\vspace{2em}

{\setstretch{1.0}
\hfill\parbox{16cm}{
\hspace*{5cm}\hspace*{-5cm}Выполнил студент группы 183, 3 курса,\\
 Соколов Павел Павлович\\

\hspace*{5cm}\hspace*{-5cm}Руководитель КР:\\
доцент Трушин Дмитрий Витальевич\\

%\hspace*{5cm}\hspace*{-5cm}Куратор:\hfill < степень>, <звание>, <ФИО полностью>\\

%\hspace*{5cm}\hspace*{-5cm}Консультант:\\
%научный сотрудник Лобачева Екатерина Максимовна\\
}
}

\vspace{\fill}

\begin{center}
Москва 2020
\end{center}

\end{titlepage}
% это титульный лист
\newpage

{
    \hypersetup{linkcolor=black}
    \tableofcontents
}

\newpage

\section{Введение}

Целью данной курсовой работы является знакомство с эллиптическими кривыми и их
применением в криптографии. В ходе её выполнения мной был повторён и изучен
необходимый алгебраический аппарат, базовые алгоритмы из несимметрической
криптографии (криптографии с открытым ключом) и был написан код на Rust с
имплементацией вышеуказанных конструкций.

\section{Необходимые определения}

\subsection{Из коммутативной алгебры}

\begin{itemize}
    \item Для \textbf{моноида} $(G, +, 0)$ выполнено:
        \begin{itemize}
            \item $G$ --- множество (не обязательно конечное);
            \item $+: G \times G \to G$ --- ассоциативная операция;
            \item $0 \in G$ нейтрален относительно $+$.
        \end{itemize}
    \item \textbf{Группа} $(G, +, 0, -)$ --- кортеж, для которого
        выполено:
        \begin{itemize}
            \item $(G, +, 0)$ --- моноид;
            \item для $-: G \to G$ выполнено $g + (-g) = 0$.
        \end{itemize}
    \item Алгебраическая структура называется \textbf{коммутативной}, если
        ассоциированная с ней операция коммутативна. Коммутативные группы обычно
        называются абелевыми.
    \item \textbf{Циклическая группа} --- группа $(G, +, 0, -)$, для которой
        существует генератор $g \in G$, т.е. для любого $h \in G$ есть
        $n \in \mathbb{N}$ такое, что $h = g \cdot n$, где $g \cdot n$ ---
        применение $+ g$ к $0$ $n$ раз.

        Несложно заметить, что циклическая группа является абелевой.
    \item \textbf{Поле} $(F, +, 0, -, \times, 1, \square^{-1})$ --- кортеж,
        обладающий следующими свойствами:
        \begin{itemize}
            \item $(F, +, 0, -)$ --- абелева группа;
            \item $0 \ne 1$;
            \item $(F, \times, 1)$ --- коммутативный моноид;
            \item $(F \setminus \{0\}, \times, 1, \square^{-1})$ --- тоже
                абелева группа (в частности, $\times$ замкнута относительно
                $F \setminus \{0\}$);
            \item $\times$ дистрибутивно относительно $+$.
        \end{itemize}
    \item \textbf{Характеристика поля} $\chr F$ --- такое $n \in \mathbb{N}$,
        что $1_F \cdot n = 0_F$.
    \item Пусть $\mathbb{F}$ --- произвольное поле. Тогда мы можем определить
        \textbf{эллиптическую кривую} как гладкую кривую в $\mathbb{F}^2$,
        которая задаётся уравнением
        \[
            y^2 + a x y + b y = x^3 + c x^2 + d x + e,
        \]
        Где коэффициенты выбираются из поля $\mathbb{F}$.

        Если $\chr F > 3$, то это уравнение с помощью замены
        координат можно привести к форме
        \[
            y^2 = x^3 + a x + b.
        \]
        Критерий гладкости: $4 a^3 + 27 b^2 \ne 0$.
    \item \textbf{Проективное пространство} над полем $\mathbb{F}$ ---
        пространство прямых некоторого линейного пространства над $\mathbb{F}$.
        Точки этого пространства можно описывать с помощью однородных координат,
        т.е. координат точек исходного пространства с определённым на них
        отношением эквивалентности $\sim$:
        \[
            \forall \lambda \in \mathbb{F} \setminus \{0\}: x \sim \lambda x.
        \]
\end{itemize}

\subsection{Из криптографии}

\begin{itemize}
    \item \textbf{Схема приватного шифрования}
        $(\verb|Gen|, \verb|Enc|, \verb|Dec|)$ --- тройка функций, где
        \begin{itemize}
            \item $\verb|Gen|: R \to K$ --- генератор приватного ключа,
            \item $\verb|Enc|: (K, R, M) \to C$ --- операция шифрования,
            \item $\verb|Dec|: (K, C) \to M \sqcup \bot$ --- операция
                дешифровки,
            \item $R$ --- множество источников случайных чисел,
            \item $K$ --- пространство приватных ключей,
            \item $M$ --- пространство передаваемых сообщений,
            \item $C$ --- пространство шифров.
        \end{itemize}
        Как следует из сигнатуры \verb|Dec|, дешифровка имеет право провалиться,
        если ей будет передан невалидный шифр.
    \item \textbf{Схема публичного шифрования}
        (\verb|Gen|, \verb|Enc|, \verb|Dec|) --- другая тройка функций. Для неё
        должно быть выполнено следующее:
        \begin{itemize}
            \item $\verb|Gen|: R \to (S, P)$ --- генератор пары
                из тайного и публичного ключа,
            \item $\verb|Enc|: (P, R, M) \to C$ --- операция шифрования,
            \item $\verb|Dec|: (S, C) \to M \sqcup \bot$ --- операция
                дешифровки,
            \item $S$ --- пространство тайных (приватных) ключей,
            \item $P$ --- пространство публичных ключей.
        \end{itemize}
        Обозначения и оговорка про провал дешифровки те же, что выше.
    \item Тройка функций $(\verb|Gen|, \verb|Encaps|, \verb|Decaps|)$ называется
        \textbf{Механизмом инкапсуляции ключа}, если
        \begin{itemize}
            \item $\verb|Gen|: R \to (S, P)$ --- генератор пары
                из ``базовых'' тайного и публичного ключа;
            \item $\verb|Encaps|: (P, R) \to (C, K)$ --- генератор сессионного
                ключа и его шифра;
            \item $\verb|Decaps|: (S, C) \to K \sqcup \bot$ --- алгоритм
                получения сессионного ключа по его шифру.
        \end{itemize}
    \item \textbf{Задача дискретного логарифмирования} формулируется так: дана
        циклическая группа $(G, +, \dots)$, её генератор $g$ и некоторый
        $h \in G$. Требуется найти $k \in \mathbb{N}$ такое, что
        $g \cdot k = h$ ($g \cdot k \vcentcolon= g + g \cdot (k - 1)$).
    \item Задача называется \textbf{сложной}, если для любого её полиномиального
        вероятностного решения вероятность правильного ответа ограничена сверху
        пренебрежимо малой относительно размера входа величиной.
    \item \textbf{Вычислительная задача Диффи-Хеллмана} (CDH): дана циклическая
        группа $(G, +, \dots)$, её генератор $g$ и некоторые $h_1, h_2 \in G$.
\end{itemize}

\newpage

\printbibliography[heading=bibintoc]

\end{document}
