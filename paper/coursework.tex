\documentclass[a4paper,14pt]{extarticle}
\usepackage{geometry}
\usepackage[T1]{fontenc}
\usepackage[utf8]{inputenc}
\usepackage[english,russian]{babel}
\usepackage{amsmath}
\usepackage{amsthm}
\usepackage{amssymb}
\usepackage{fancyhdr}
\usepackage{setspace}
\usepackage{graphicx}
\usepackage{colortbl}
\usepackage{tikz}
\usepackage{pgf}
\usepackage{subcaption}
\usepackage{listings}
\usepackage{indentfirst}
\usepackage[colorlinks,citecolor=blue,linkcolor=blue,bookmarks=false,hypertexnames=true, urlcolor=blue]{hyperref}
\usepackage{indentfirst}
\usepackage{mathtools}
\usepackage{booktabs}
\usepackage[flushleft]{threeparttable}
\usepackage{tablefootnote}

\usepackage{chngcntr} % нумерация графиков и таблиц по секциям
\counterwithin{table}{section}
\counterwithin{figure}{section}

\usepackage[backend=biber, citestyle=authoryear]{biblatex}
\addbibresource{bibliography.bib}
\DeclareFieldFormat{labelnumberwidth}{#1\adddot}
\setlength{\biblabelsep}{5pt}

\makeatletter
\renewcommand{\@biblabel}[1]{#1.} % Заменяем библиографию с квадратных скобок на точку:
\makeatother

\geometry{left=1.5cm}% левое поле
\geometry{right=1.5cm}% правое поле
\geometry{top=1.5cm}% верхнее поле
\geometry{bottom=1.5cm}% нижнее поле

\renewcommand{\theenumi}{\arabic{enumi}}% Меняем везде перечисления на цифра.цифра
\renewcommand{\labelenumi}{\arabic{enumi}}% Меняем везде перечисления на цифра.цифра
\renewcommand{\theenumii}{.\arabic{enumii}}% Меняем везде перечисления на цифра.цифра
\renewcommand{\labelenumii}{\arabic{enumi}.\arabic{enumii}.}% Меняем везде перечисления на цифра.цифра
\renewcommand{\theenumiii}{.\arabic{enumiii}}% Меняем везде перечисления на цифра.цифра
\renewcommand{\labelenumiii}{\arabic{enumi}.\arabic{enumii}.\arabic{enumiii}.}% Меняем везде перечисления на цифра.цифра

\DeclareMathOperator{\chr}{char}

\newcommand{\deff}{\vspace{0.2cm} \noindent}

\begin{document}
\begin{titlepage}
\newpage

{\setstretch{1.0}
\begin{center}
Федеральное государственное автономное образовательное учреждение высшего образования «Национальный исследовательский университет «Высшая школа экономики»
\\
\bigskip
Факультет компьютерных наук \\
Основная образовательная программа \\
Прикладная математика и информатика \\
\end{center}
}

\vspace{8em}

\begin{center}
{\Large КУРСОВАЯ РАБОТА}\\
\textsc{\textbf{
Исследовательский проект на тему
\linebreak
"Коммутативная алгебра"}}
\end{center}

\vspace{2em}

{\setstretch{1.0}
\hfill\parbox{16cm}{
\hspace*{5cm}\hspace*{-5cm}Выполнил студент группы 183, 3 курса,\\
 Соколов Павел Павлович\\

\hspace*{5cm}\hspace*{-5cm}Руководитель КР:\\
доцент Трушин Дмитрий Витальевич\\

%\hspace*{5cm}\hspace*{-5cm}Куратор:\hfill < степень>, <звание>, <ФИО полностью>\\

%\hspace*{5cm}\hspace*{-5cm}Консультант:\\
%научный сотрудник Лобачева Екатерина Максимовна\\
}
}

\vspace{\fill}

\begin{center}
Москва 2020
\end{center}

\end{titlepage}
% это титульный лист
\newpage

{
    \hypersetup{linkcolor=black}
    \tableofcontents
}

\newpage

\section{Введение}

Целью данной курсовой работы является знакомство с эллиптическими кривыми и их
применением в криптографии. В ходе её выполнения мной был повторён и изучен
необходимый алгебраический аппарат, базовые алгоритмы из несимметрической
криптографии (криптографии с открытым ключом) и был написан код на Rust с
имплементацией (де)шифратора на основе эллиптических кривых. Код доступен по
ссылке \url{https://github.com/TurtlePU/elliptic}.

\newpage

\section{Необходимые определения}

\subsection{Из коммутативной алгебры}

\noindent \textbf{Моноид} $(G, +, 0)$ --- любой кортеж, для которого выполнено:

$\bullet$ $G$ --- множество (не обязательно конечное);

$\bullet$ $+: G \times G \to G$ --- ассоциативная операция;

$\bullet$ $0 \in G$ нейтрален относительно $+$.

\deff \textbf{Группа} $(G, +, 0, -)$ --- кортеж, для которого выполено:

$\bullet$ $(G, +, 0)$ --- моноид;

$\bullet$ для $-: G \to G$ выполнено $g + (-g) = 0$.

\deff У \textbf{коммутативной} алгебраической структуры её операция
коммутативна. Коммутативные группы обычно называются абелевыми.

\deff \textbf{Циклическая группа} --- группа $(G, +, 0, -)$, для которой
существует генератор $g \in G$, т.е. для любого $h \in G$ есть
$n \in \mathbb{N}$ такое, что $h = g \cdot n$, где $g \cdot n$ --- применение
$+ g$ к $0$ $n$ раз.

Несложно заметить, что циклическая группа является абелевой.

\deff \textbf{Поле} $(F, +, 0, -, \times, 1, \square^{-1})$ --- кортеж,
обладающий следующими свойствами:

$\bullet$ $(F, +, 0, -)$ --- абелева группа;

$\bullet$ $0 \ne 1$;

$\bullet$ $(F, \times, 1)$ --- коммутативный моноид;

$\bullet$ $(F \setminus \{0\}, \times, 1, \square^{-1})$ --- тоже абелева группа
(в частности, $\times$ замкнута относительно $F \setminus \{0\}$);

$\bullet$ $\times$ дистрибутивно относительно $+$.

\deff \textbf{Характеристика поля} $\chr F$ --- такое $n \in \mathbb{N}$, что
$1_F \cdot n = 0_F$.

\deff \textbf{Эллиптическая кривая} над полем $\mathbb{F}$ --- гладкая кривая в
$\mathbb{F}^2$, которая задаётся уравнением

\[
    y^2 + a x y + b y = x^3 + c x^2 + d x + e,
\]

Где коэффициенты выбираются из поля $\mathbb{F}$.

Если $\chr F > 3$, то это уравнение с помощью замены координат можно привести к
форме

\[
    y^2 = x^3 + a x + b.
\]

Критерий гладкости: $4 a^3 + 27 b^2 \ne 0$.

\deff \textbf{Проективное пространство} над полем $\mathbb{F}$ ---
пространство прямых некоторого линейного пространства над $\mathbb{F}$. Точки
этого пространства можно описывать с помощью однородных координат, т.е.
координат точек исходного пространства с определённым на них отношением
эквивалентности $\sim$:

\[
    \forall \lambda \in \mathbb{F} \setminus \{0\}: x \sim \lambda x.
\]

\newpage

\subsection{Из криптографии}

\noindent \textbf{Схема приватного шифрования}
$(\verb|Gen|, \verb|Enc|, \verb|Dec|)$ --- тройка функций, где

$\verb|Gen|: (R, \mathbb{N}) \to K$ --- генератор приватного ключа,

$\verb|Enc|: (K, R, M) \to C$ --- операция шифрования,

$\verb|Dec|: (K, C) \to M \sqcup \bot$ --- операция дешифровки,

$R$ --- множество источников (псевдо)случайных чисел,

$K$ --- пространство приватных ключей,

$M$ --- пространство передаваемых сообщений,

$C$ --- пространство шифров.

\noindent Как следует из сигнатуры \verb|Dec|, дешифровка имеет право
провалиться, если ей будет передан невалидный шифр.

\deff \textbf{Схема публичного шифрования} (\verb|Gen|, \verb|Enc|, \verb|Dec|)
--- другая тройка функций. Для неё должно быть выполнено следующее:

$\verb|Gen|: (R, \mathbb{N}) \to (S, P)$ --- генератор пары из тайного и
публичного ключа,

$\verb|Enc|: (P, R, M) \to C$ --- операция шифрования,

$\verb|Dec|: (S, C) \to M \sqcup \bot$ --- операция дешифровки,

$S$ --- пространство тайных (приватных) ключей,

$P$ --- пространство публичных ключей.

\noindent Обозначения и оговорка про провал дешифровки те же, что выше.

\deff \textbf{Механизм инкапсуляции ключа}
$(\verb|Gen|, \verb|Encaps|, \verb|Decaps|)$ удовлетворяет такому интерфейсу:

$\verb|Gen|: R \to (S, P)$ --- генератор пары из ``базовых'' тайного и
публичного ключа;

$\verb|Encaps|: (P, R) \to (C, K)$ --- генератор сессионного ключа и
его шифра;

$\verb|Decaps|: (S, C) \to K \sqcup \bot$ --- алгоритм получения ключа по его
шифру.

\deff \textbf{Задача дискретного логарифмирования} формулируется так: дана
циклическая группа $(G, +, \dots)$, её генератор $g$ и некоторый
$h \in G$. Требуется найти $k \in \mathbb{N}$ такое, что
$g \cdot k = h$.

Задача дискретного логарифмирования называется
\textbf{сложной относительно} $G$, если для любого её полиномиального
вероятностного решения вероятность правильного ответа ограничена сверху
пренебрежимо малой по сравнению с $\log(|G|)$ величиной.

\deff \textbf{Вычислительная задача Диффи-Хеллмана} (CDH): дана циклическая
группа $(G, +, \dots)$, её генератор $g$ и некоторые
$g \cdot a, g \cdot b \in G$. Требуется найти $g \cdot (a \cdot b)$.

Определение \textbf{сложности относительно} $G$ аналогично дискретному
логарифмированию.

\deff \textbf{Распознавательная задача Диффи-Хеллмана} (Decisional
Diffie-Hellman, DDH) состоит в отличении случайно выбранного элемента
$G$ от результата CDH. Другими словами, даны случайные $g \cdot a$,
$g \cdot b$ и $h$. Нужно ответить, правда ли, что
$h = g \cdot (a \cdot b)$, или $h$ тоже был выбран случайно.

Пусть $\mathcal{G}(r, n)$ --- алгоритм генерации случайной группы $G$
такой, что $2^n \le |G| < 2^{n + 1}$ ($r$ --- источник случайности).
Тогда DDH называется \textbf{сложной относительно} $\mathcal{G}$, если
для любого вероятностного алгоритма $\mathcal{A}$ выполнено

\begin{multline*}
    \bigg| P \Big[
        \mathcal{A}(G, |G|, g, g \cdot x, g \cdot y, g \cdot z) = 1
    \Big] -\\- P \Big[ \mathcal{A}(
        G, |G|, g, g \cdot x, g \cdot y, g \cdot (x \cdot y)
    ) = 1 \Big] \bigg| \le \mathrm{negl}(n),
\end{multline*}

где $G$ и $g$ генерируются $\mathcal{G}$, $x, y, z$
выбираются случайно и равномерно, а $\mathrm{negl}(n)$ --- пренебрежимо
малая по сравнению с $n$ величина.

\deff \textbf{Chosen plaintext attack} --- атака на схему шифрования, в ходе
которой злоумышленник может выбирать, какие сообщения будут посланы
через канал связи, и читать получившиеся шифры. Схемы, устойчивые к
таким атакам, называются \textbf{CPA-secure}.

\deff \textbf{Chosen ciphertext attack} --- усиленная версия CPA-атаки. В
рамках CCA-атаки злоумышленник может также запрашивать расшифровку
произвольных шифров. Схемы, устойчивые к таким атакам, называются
\textbf{CCA-secure}.

\deff \textbf{Двусторонняя функция} --- такая функция, что и вычисление
образа, и вычисление прообраза имеют близкую асимптотическую сложность.

\deff \textbf{Односторонняя функция} --- такая функция, для которой
вычисление прообраза асимптотически гораздо сложнее вычисления образа.

\newpage

\section{Эллиптические кривые и группы}
\label{elliptic}

Возьмём произвольное поле $\mathbb{F}$ с $\chr \mathbb{F} > 3$. Добавим в
$\mathbb{F}^2$ точку ``на бесконечности'' $\mathcal{O}$. Во-первых, мы таким
образом получим проективное пространство над $\mathbb{F}^3$; во-вторых, точки на
произвольной эллиптической кривой вместе с $\mathcal{O}$ образуют группу. Если
конкретно, $\mathcal{O}$ --- ноль группы, обратная точка --- отражение точки
относительно оси $0x$, а операция группы $p + q$ определена так:

\vspace{0.2cm}

\textbf{1.} Если одна из точек --- бесконечная, ответ очевиден.

\textbf{2.} Если $p = q$, проведём касательную к кривой в точке $p = q$.
Полученная прямая либо параллельна $0y$ (и тогда $p + q = \mathcal{O}$),
либо пересекает кривую в ещё одной точке $r$.

\textbf{3.} Иначе проведём прямую через $p$ и $q$. Она либо касается кривой в
одной из указанных точек, назовём её $r$, либо пересекается с кривой ещё
в одной точке $r$.

\vspace{0.2cm}

По определению $p + q \vcentcolon= -r$. Заметим, что в рассуждениях выше мы
делали достаточно сильные утверждения про геометрию эллиптических кривых; это,
если верить \cite{textbook}, легко показать.

Кроме того, в \cite{textbook} указаны алгебраические формулы для вычисления
$p + q$: пусть $p = (x_p, y_p)$, $q = (x_q, y_q)$ (иначе ответ очевиден).

\vspace{0.2cm}

\textbf{1.} Если $x_p \ne x_q$, тогда $p + q = (x_r, y_r)$, где

\[
    \begin{array}{rcl}
        m &=& \frac{y_q - y_p}{x_q - x_p}; \\
        x_r &=& m^2 - x_p - x_q; \\
        y_r &=& m \cdot (x_p - x_r) - y_p.
    \end{array}
\]

\textbf{2.} Если $x_p = x_q$, но $y_p \ne y_q$, то $p = -q$, так что
$p + q = \mathcal{O}$.

\textbf{3.} Если $p = q$ и $y_p = y_q = 0$, то $p + q = \mathcal{O}$.

\textbf{4.} Если $p = q = (x, y)$, но $y \ne 0$, то $2 p = (x', y')$, где

\[
    \begin{array}{rcl}
        y^2 &=& x^3 + a x + b \text{ --- уравнение кривой}; \\
        m &=& \frac{3 x^2 + a}{2 y}; \\
        x' &=& m^2 - 2 x; \\
        y' &=& m \cdot (x - x') - y.
    \end{array}
\]

\newpage

\section{Некоторые криптографические алгоритмы на основе эллиптических кривых}

\subsection{Схема публичного шифрования Эль-Гамаля}

На самом деле, схема Эль-Гамаля может использовать произвольный алгоритм
генерации группы $\mathcal{G}$. Сама схема выглядит следующим образом:

\vspace{0.2cm}

$\verb|Gen|: (R, \mathbb{N}) \to (S, P)$: запустить $\mathcal{G}$, получить
группу $G$, её порядок $q$ и генератор $g$. Выбрать случайный
$x \in \mathbb{Z}_q$, вычислить $h \vcentcolon= g \cdot x$. Публичный ключ ---
$(G, q, g, h)$, приватный --- $(G, q, g, x)$. Пространство сообщений --- $G$.

$\verb|Enc|: (P, R, G) \to C$: выбрать случайное $y \in \mathbb{Z}_q$,
шифром будет $(g \cdot y, h \cdot y + m)$, где $m$ --- исходное
сообщение.

$\verb|Dec|: (S, (G, G)) \to G$: пусть $(c_1, c_2)$ --- переданный
шифр. Было зашифровано $c_2 - c_1 \cdot x$.

\vspace{0.2cm}

При условии сложности DDH относительно $\mathcal{G}$ схема Эль-Гамаля устойчива
к CPA-атакам \cite{textbook}.

\subsection{Механизм инкапсуляции ключа Эль-Гамаля}

Аналогично предыдущему пункту, можно использовать любой $\mathcal{G}$, не только
для эллиптических кривых.

Итак, механизм инкапсуляции ключа a-la Эль-Гамаль:

\vspace{0.2cm}

$\verb|Gen|: (R, \mathbb{N}) \to (S, P)$: запустить $\mathcal{G}$ на входных
аргументах, получить группу $G$, её порядок $q$ и генератор $g$. Выбрать
случайный $x \in \mathbb{Z}_q$, вычислить $h \vcentcolon= g \cdot x$. Задать
функцию $H: G \to \{0, 1\}^{l(n)}$, где $l$ является деталью реализации.
Публичный ключ --- $(G, q, g, h, H)$, приватный --- $(G, q, g, x, H)$.

$\verb|Encaps|: (P, R) \to (C, K)$. Выберем случайный $y \in \mathbb{Z}_q$.
Шифром будет $g \cdot y$, ключом --- $H(h \cdot y)$.

$\verb|Decaps|: (S, C) \to M$: для шифра $c$ ключом является $H(c \cdot x)$.

\newpage

\section{От алгоритмов к шифрованию}

В предыдущей секции мы рассмотрели алгоритмы, позволяющие шифровать $O(1)$
элементов некоторой группы $G$ и инкапсулировать ключи. Однако на практике нам
1) нужно шифровать бинарные сообщения произвольной длины; 2) не нужно
инкапсулировать какие-то там ключи. В данной секции мы заполним пробелы и
покажем, как компонировать эти алгоритмы.

\subsection{Шифрование потока данных}
\label{stream}

Допустим, у нас есть схема шифрования одного элемента некой группы $G$ с помощью
элементов множества $M$. Для начала заметим, что тогда у нас есть схема
шифрования $\lfloor \log(|G|) \rfloor$ бит с помощью $\lceil \log(|M|) \rceil$
бит:

\vspace{0.2cm}

$\bullet$ При шифровке отобразим шифруемые биты в группу $G$ заранее
зафиксированной двусторонней инъекцией; полученный шифр можно инъективно
отобразить в $\{0, 1\}^{\lceil \log(|M|) \rceil}$.

$\bullet$ При дешифровке нужно сначала отобразить биты обратно в $M$ (если у бит
нет прообраза, вернуть $\bot$); после дешифровки нужно отобразить
$g \in G$ обратно в $\{0, 1\}^{\lfloor \log(|G|) \rfloor}$ (если у $g$
нет прообраза, вернуть $\bot$).

\vspace{0.2cm}

Очевидно, что устойчивость схемы шифрования от этого не меняется, поскольку
односторонняя функция шифрования компонировалась с двусторонними энкодингами.

Далее, если у нас есть схема шифрования $k_1$ бит с помощью $k_2$ бит, у нас
есть схема шифрования $n k_1$ бит с помощью $n k_2$ для любых
$n, k \in \mathbb{N}$: нужно просто разбить шифруемый поток на чанки длины
$k_1$, зашифровать каждый, сконкатенировать; для дешифровки алгоритм аналогичен.

Но это очевидно; интересно то, что при этом устойчивость схемы шифрования тоже
не меняется, как показано в \cite{textbook}.

Таким образом, алгоритмов из предыдущей секции достаточно, чтобы шифровать
произвольные последовательности.

\newpage

\subsection{Гибридная схема шифрования}
\label{hybrid}

Механизмы инкапсуляции ключа были введены не просто так: это базовый блок для
\textbf{гибридной схемы шифрования}, основная идея которой заключается в том,
чтобы шифровать каждое сообщение своим ключом, который передаётся вместе с
сообщением (само собой, ключ тоже зашифрован).

Более формально, пусть $(\verb|Gen|', \verb|Encaps|, \verb|Decaps|)$ ---
механизм инкапсуляции ключа некоторой приватной схемы шифрования
$(\_, \verb|Enc|', \verb|Dec|')$. Тогда можно построить
\textbf{гибридную схему публичного шифрования}:

\vspace{0.2cm}

$\verb|Gen| = \verb|Gen|'$;

$\verb|Enc|: (P, R, M) \to C$: запустим \verb|Encaps| на $(P, R)$,
получим $(c_1, k)$. С помощью $\verb|Enc|'$ и $k$ зашифруем сообщение,
получим шифр $c_2$. Итоговым шифром будет $(c_1, c_2)$.

$\verb|Dec|: (S, (C_1, C_2)) \to M \sqcup \bot$: сначала попытаемся
получить ключ с помощью $\verb|Decaps|(S, C_1)$. В случае успеха
попытаемся расшифровать $C_2$ полученным ключом.

\vspace{0.2cm}

Одной из самых известных гибридных схем шифрования является ECIES, Elliptic
Curve Integrated Encryption Scheme. В ней в качестве механизма инкапсуляции
используется механизм Эль-Гамаля. Причём, в отличие от публичной схемы
шифрования Эль-Гамаля, ECIES устойчива к CCA-атакам \cite{textbook}.

\newpage

\section{Тонкости реализации}

\subsection{Однородные координаты}

Как правило, вычисление обратного по умножению в конечном поле --- затратная по
времени операция. И она, как можно заметить в разделе \hyperref[elliptic]{3},
используется при сложении двух точек на эллиптической кривой. К счастью, точки
находятся в проективном пространстве, так что мы можем использовать однородные
координаты и
избежать деления почти полностью (в конце всё равно нужно будет привести точки к
каноническому виду, потому что иначе злоумышленник может получить дополнительную
информацию о производившихся в процессе шифрования действиях). Если быть точным,
формулы приобретают следующий вид: пусть $p = (x_p, y_p, z_p)$,
$q = (x_q, y_q, z_q)$; тогда

\begin{itemize}
    \item Если $p \ne q$, $p + q = (x, y, z)$, где
        \[
            \begin{array}{rcl}
                u &=& y_q z_p - y_p z_q; \\
                v &=& x_q z_p - x_p z_q; \\
                w &=& u^2 z_p z_q - v^3 - 2 v^2 x_p z_q; \\
                q &=& v^3 y_p z_q; \\
                x &=& v w; \\
                z &=& z_p z_q v^3; \\
                y &=& u \cdot (v^2 x_p z_q - w) - q.
            \end{array}
        \]
    \item Если $p = q$ и $p \ne -q$, то $p + q = (x, y, z)$, где
        \[
            \begin{array}{rcl}
                y^2 &=& x^3 + a x + b \text{ --- уравнение кривой}; \\
                q &=& 2 y z; \\
                n &=& 3 x^2 + a z^2; \\
                p &=& 4 x y^2 z; \\
                u &=& n^2 - 2 p; \\

                x &=& u q; \\
                z &=& q^3; \\
                y &=& n \cdot (p - u) - 8 y^4 z^2.
            \end{array}
        \]
\end{itemize}

\subsection{Генерация групп на эллиптических кривых}
\label{generator}

Схема Эль-Гамаля предполагает генерацию случайного генератора группы. По теореме
Лагранжа, если взять эллиптическую кривую, которая проходит через $p - 1$ точку,
любая точка на кривой будет генератором. Таким образом, достаточно научиться
выбирать случайную точку на кривой. Но есть одна загвоздка: для этого нужно либо
выбирать случайный $y$ и решать кубическое уравнение на $x$ в произвольном поле,
либо выбирать случайный $x$ и вычислять корень из $x^3 + a x + b$. К счастью, в
\cite{sqrt} приведены детерминированный алгоритм вычисления корня для некоторых
специальных полей и вероятностный алгоритм для вычисления корня в произвольном
поле. В нашей имплементации реализован тривиальный случай для $\mathbb{Z}_p$
при условии $p = 3 \mod \;4$: в этом случае для вычисления $\sqrt{x}$ достаточно
вычислить $x^{(p + 1) / 4}$ и проверить, является ли это решением.

Другой, менее честный вариант: аналитически найти какую-нибудь одну точку $g$ на
кривой. Если порядок группы $p$ простой, она будет генератором, и все её степени
(кроме $\mathcal{O}$) тоже будут генераторами. Выберем случайную степень
$x \in \{1,\dots,p - 1\}$; $g \cdot x$ будет нашим случайным генератором.

\subsection{Энкодинг бит точками на кривой}

В секции \hyperref[stream]{5.1} мы обсудили, что для кодирования произвольного
текста с помощью
элементов группы достаточно ввести двустороннюю инъекцию из пространства букв в
элементы группы. В \cite{encoding} предложен неплохой алгоритм, который, однако,
работает только для полей вида $\mathbb{Z}_q$. Если кратко, алгоритм состоит в следующем:

\begin{itemize}
    \item Выделить в $\mathbb{Z}_q$ бакеты, в каждом из которых есть $x$, для
        которого есть $y$ такой, что $y^2 = x^3 + a x + b$;
    \item Поставить каждой букве в соответствие бакет, при энкодинге находить
        $x$ линейным поиском внутри бакета, при декодинге определять букву по
        бакету. Заметим, что здесь опять нужно вычисление корня в поле; алгоритм
        мы уже обсудили в секции \hyperref[generator]{6.2}.
\end{itemize}

Структура $\mathbb{Z}_q$ здесь используется только для того, чтобы соответствие
бакета букве вычислялось простыми линейными функциями. Есть основания полагать,
что существует эффективное обобщение алгоритма на произвольное поле.

В ходе работы над (де)шифратором были попытки использовать этот алгоритм, но
подобрать подходящий (и не слишком большой) размер бакета мне не удалось. Так
что используется наивный алгоритм: чтобы закодировать байт $x$, нужно возвести
генератор группы в степень $x$, а для раскодировки забрутфорсить все возможные
значения $x$.

\newpage

\section{Результаты вычислительных экспериментов}

В качестве proof-of-concept всего приведённого выше теоретического материала
были проведены два вычислительных эксперимента:

\begin{enumerate}
    \item Проверить корректность работы (де)шифраторов по схеме Эль-Гамаля на
        основе используемой на практике эллиптической кривой P224, а также на
        основе циклической подгруппы $\mathbb{Z}_p^*$ соответствующего порядка
        (MODP2048).
    \item Сравнить два алгоритма взлома шифрования Эль-Гамаля: брутфорс для
        эллиптической кривой и алгоритм Полларда для циклической подгруппы
        $\mathbb{Z}_p^*$.
\end{enumerate}

\subsection{Корректность (де)шифровки}

В данном эксперименте проверялась корректность алгоритмов (де)шифровки с помощью
получения случайного входа, его шифровки, дешифровки полученного шифра и
сравнения результата со входом.

Неудивительно, что (де)шифраторы отработали безошибочно, однако хочется заметить
следующее:

\begin{itemize}
    \item Время работы программы ощутимо больше, чем у готовых решений;
    \item Длина шифротекста на порядки больше, чем длина шифруемого сообщения.
\end{itemize}

Конечно, это происходит в первую очередь из-за избыточности энкодинга: каждый
байт кодируется несколькими словами длиной более двухсот бит (с циклической
подгруппой $\mathbb{Z}_p^*$ всё ещё хуже, ведь энкодинг там приходится делать
точно такой же, а элементы группы занимают ещё больше памяти). Существует два
одинаково неправильных варианта решения проблемы:

\begin{itemize}
    \item Кодировать одним элементом группы больше одного байта. Это сокращает
        длину шифротекста, но сложность декодинга при этом растёт
        экспоненциально.
    \item Взять группу меньшего порядка. Это и сократит длину шифротекста, и
        уменьшит сложность всех алгоритмов, но при этом потеряется
        криптографическая устойчивость: если порядок группы мал, даже наивные
        алгоритмы взлома завершат свою работу достаточно быстро.
\end{itemize}

Таким образом, главный результат следующий: шифрование на основе эллиптических
кривых (как и на основе циклических подгрупп $\mathbb{Z}_p^*$) лучше всего
подходит для гибридных схем шифрования (секция \hyperref[hybrid]{5.2}). Для работы с собственно
текстом лучше подходят группы в полях многочленов.

\subsection{Сравнение алгоритмов взлома}

Целью данного эксперимента является демонстрация эффективности групп на основе
эллиптических кривых по сравнению с циклическими подгруппами $\mathbb{Z}_p^*$.
Эффективность с алгоритмической точки зрения заключается в следующем: в то время
как для циклических подгрупп $\mathbb{Z}_p^*$ существуют сравнительно
эффективные алгоритмы поиска дискретного логарифма, для эллиптических кривых
алгоритмы лучше брутфорса пока не существуют. Следовательно, для достижения той
же безопасности, что у эллиптической кривой, $p$ в $\mathbb{Z}_p^*$ нужно
подбирать гораздо больше. В \cite{textbook} приведена таблица с явными соответствиями
параметров.

Для практической проверки этого теоретического результата использовать те же
группы, что в предыдущем эксперименте, не подойдёт: длина искомых параметров
больше двухсот бит, так что брутфорс займёт больше времени, чем было дано на
выполнение этой работы. Так что в ходе эксперимента было решено подобрать
эллиптическую кривую с малым порядком группы, а $p$ подобрать так, чтобы время
работы двух алгоритмов было близким.

Таким образом для сравнения были выбраны кривая $y^2 = x^3 + x + 1$ над полем
$\mathbb{Z}_{127}$ с генератором $(4, 14)$ и группа $\mathbb{Z}_{3257}^*$.
Брутфорс на кривой для всех точек занимает примерно столько же времени (4.5
секунды), сколько алгоритм Полларда на группе $\mathbb{Z}_{3257}^*$.

\newpage

\printbibliography[heading=bibintoc]

\end{document}
